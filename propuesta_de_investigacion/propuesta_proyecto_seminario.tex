\documentclass[12pt,letterpaper]{article}
\usepackage[utf8]{inputenc}
\usepackage[T1]{fontenc}
\usepackage[spanish]{babel}
\usepackage{amsmath}
\usepackage{amsfonts}
\usepackage{amssymb}
\usepackage{graphicx}
\usepackage{enumerate}
\usepackage[left=2.54cm, right=2.54cm, top=2.54cm, bottom=2.54cm]{geometry}
%\usepackage{apalike}
\author{J. Moisés Arias}
\title{Propuesta de Proyecto de Investigación para el Seminario en Física de Radiaciones}

\AtBeginDocument{\renewcommand{\refname}{Fuentes bibliográficas}}
\begin{document}
	\maketitle
	\section*{Información general del proyecto}
		\paragraph{Título del proyecto:} Determinación de un modelo de supervivencia celular en tejidos cancerosos ante la incidencia de radiación como tratamiento terapeútico.
		\paragraph{Responsable:} José Moisés Arias Núñez
		\paragraph{Dirección electrónica:} jmarias@unah.hn
		\paragraph{No. de cuenta estudiantil:} 20141003876
		\paragraph{Centro regional universitario:} Ciudad Universitaria - José Trinidad Reyes
		\paragraph{Departamento:} Gravitación, Radiaciones y Altas Energías
		\paragraph{Carrera:} Licenciatura en Física
		\paragraph{Duración del proyecto:} 3 meses
		\paragraph{Área del conocimiento del proyecto:} Radiobiología
	\section*{Descripción del proyecto}
		\subsection*{Resumen del proyecto}
		El problema a mano consiste en encontrar un modelo (o varios, de ser posible) que describan la razón de células supervivientes frente a las células afectadas por radioterapia en un tejido canceroso. 
		
		Se plantea resolver el problema a partir de los principios físicos de la interacción de las radiaciones con la materia tomando en cuenta la deposición de energía de dichas radiaciones en el tejido, también llamado dosis de la radiación, y la medida en que el tejido responde a esa dosis. Tales principios corresponden a leyes y teoremas propios de áreas como la \textit{Física de Partículas}, el \textit{Electromagnetismo} y la \textit{Termodinámica}. 
		
		En estos marcos de trabajo de la Física encontramos el formalismo que describe la transferencia de energía de las partículas constituyentes de un tipo de radiación a través de colisiones de dichas con la materia, la densidad energética de un haz de radiación electromagnética y su tasa de flujo desde la fuente y la influencia de las variables de estado del sistema como su volumen, temperatura y cantidad materia en la medida en que dicha energía radiativa es absorbida por el tejido. 
		\subsection*{Planteamiento del problema}
		La tasa de supervivencia de las células en un tejido ante un tratamiento terapeútico con radiación es de suma importancia en la determinación de la eficiencia, calidad y seguridad del tratamiento. Pues con la incidencia de la radiación se busca destruir el tejido enfermo antes de que se propague y cuando eso no es posible, reducir el volumen afectado para menguar los síntomas dolorosos que aquejan a un paciente (estos últimos constituyen tratamientos paleativos). 
		
		Muchos factores pueden influir en la respuesta de un tejido al tratamiento, uno de ellos es la concentración de oxígeno en la muestra y la distribución de las células enfermas entre las células sanas. El tratamiento no es selectivo y ataca por igual a ambas células, por lo que queremos que el costo de destruir tejido \textit{bueno} venga con una reducción significativa del tejido \textit{enfermo}. 
		
		Algunas preguntas que se pueden hacer son:
		
		\begin{enumerate}
			\item ¿Cómo influye el tratamiento a nivel genético? 
			\item ¿Qué influencia tiene la radiosensibilidad del tejido al tratamiento? 
			\item ¿Cómo podemos cuantificar esa sensibilidad a la radiación como respuesta a un estímulo? 
			\item ¿Existe una forma de modificar esa sensibilidad? 
			\item ¿Cómo se traducen los efectos moleculares en celulares y respuestas tisulares? 
			\item ¿Qué factores están detrás de la capacidad de respuesta de las células a este tipo de tratamiento?
			\item ¿Qué efectos surgen en el tejido irradiado y el tejido vecino después del tratamiento? 
			\item ¿Cuáles de esos efectos son deterministas y cuáles son estocásticos?
			\item ¿Hay efectos a largo plazo? ¿Cuáles son sus implicaciones? 
			\item ¿Qué particularidades podemos esperar del tratamiento en pacientes especiales (gestantes o pediátricos)?
		\end{enumerate}
		\subsection*{Estado del arte de la invesigación}
		La literatura comprende tratados extensos bajo la luz de las Ciencias Médicas y Biológicas, mas los elementos Físico-Matemáticos suelen ser tratados como un complemento en el tratamiento de las observaciones experimentales. Un tratamiento teórico formal y riguroso partiendo de principios y leyes fundamentales sobre la naturaleza (tales como las que ofrece la Física) es virtualmente escaso. 
		
		La Estadística Descriptiva toma un papel protagónico en el \textit{set de herramientas} propuesto por la documentación clásica, pero rara vez se emplea en la búsqueda de modelos semi-empíricos que permitan hacer predicciones de los fenómenos inherentes a la interacciónd e la radiación con la materia viva. 
		
		Las publicaciones accesibles vienen en la forma de anales o recopilaciones de artículos en congresos o series de publicaciones internacionales. En la literatura científica hondureña no se encuentran tales registros, si los hay, son escasos o han tenido poco impacto. 
		
		A nivel latinoamericano existen investigaciones de carácter clínico procedentes de Cuba, Chile y Colombia en las que se consideran las tazas de supervivencia celular en incidencias específicas de cáncer. 
		
		Estos artículos indagan sobre las condiciones fisiológicas que agravan las dificultades del tratamiento en el sentido de obtener una respuesta signifivativa en el paciente ante el tratamiento, más el elemento cuantitativo de la cuestión, cuando existe, procede de un análisis estadístico descriptivo auxiliar al discurso médico clínico. No obstante, un cálculo del modelo semiempírico o teórico de la supervivencia celular es inexistente. 
		
		\subsection*{Objetivos del proyecto}
			\subsubsection*{Objetivo general}
			\begin{itemize}
				\item Determinar un modelo teórico desde primeros principios Físicos y Matemáticos sobre la supervivencia celular en tejidos enfermos y sanos ante la interacción con una fuente radiactiva en tratamientos terapeúticos contra el cáncer en general.
			\end{itemize}
			\subsubsection*{Objetivos específicos}
				\begin{itemize}
					\item Emplear métodos matemáticos propios de la Física y el análisis estadístico en la obtención de un modelo teórico de supervivencia celular en radioterapias. 
					\item Implementar principios termodinámicos, electrodinámicos, nucleares y mecánico-cuánticos para la descripción cuantitativa de la respuesta celular y tisular a la radiación. 
					\item Simular computacionalmente las respuestas de un tejido a la radiación mediante la evaluación funcional del modelo. 
					\item Comparar los resultados teóricos con los registros experimentales disponibles en la literatura biomédica. 
				\end{itemize}
			
			\subsection*{Metodología}
			Se planea hacer una revisión exhaustiva de la literatura actual en una secuencia cronológica para evaluar la transición de la teoría hacia los marcos de trabajo actuales; la intención es encontrar qué acercamientos se han realizado antes, qué les ha reemplazado y por qué; ¿existe algo rescatable de esos acercamientos pasados?
			
			Revisar la literatura biomédica de los casos en donde un análisis radiobiológico cuantitativo ha sido empleado y evaluar el nivel de formalidad y rigor matemático con el que los datos han sido tratados y hasta qué grado se alcanzó un modelado del fenómeno, si se intentó hacerlo. 
			
			Construir una simulación computacional que evalúe el o los modelos de la investigación y los compararlos con los existentes en la literatura y determinar su fidelidad a los datos presentados en la literatura biomédica. 
			
			\subsection*{Resultados, productos e impactos esperados}
			Principalmente se quiere establecer un modelo matemático confiable apoyado por las teorías fundamentales de la Física que subyacen a los fenómenos detrás de la interacción de la radiación con la materia (viva). 
			
			Se desea elaborar una simulación numérica que permita predecir la respuesta de un paciente de cáncer ante un tratamiento radiactivo y que esta simulación sirva como una herramienta en la planificación del tratamiento y el control de calidad del mismo. 
			
			Se desea publicar estos resultados para contribuir a la cultura científica nacional en el marco de la radiobiología y la Física de Radiaciones aplicada. De manera que se pueda contar con un antecedente de esta línea de trabajo que conduzca a futuras indagaciones. 
			
			\subsection*{Cronograma}
			\paragraph{Fase 1: Revisión de la literatura 1} En esta fase se realizará una lectura extensa de la literatura radiobiológica, biomédica y físico-médica sobre la cuestión particular de cómo responde el tejido vivo ante la radiación, las condiciones que subyacen a tales respuestas y los modelos matemáticos propuestos para la descripción cuantitativa de estos fenómenos.  
			\paragraph{Fase 2: Revisión de la literatura 2} En esta fase se revisarán los principios físicos que pueden aportar una sustentación teórica a los modelos existentes. Se evaluará qué aspectos de las teorías físicas más compatibles con la cuestión planteada pueden contribuir a resolver el problema.
			\paragraph{Fase 3: Deducción del modelo o modelos} Partiendo de los principios revisados en la fase 2, se indagará sobre la construcción de una ecuación que relacione las variables físicas importantes con las respuestas observadas en el ejericio biomédico. 
			\paragraph{Fase 4: Programación de la simulación numérica} Tras la obtención del modelo matemático se busca crear una simulación numérica eficiente que devuelva predicciones sobre el comportamiento de un tejido vivo ante diferentes estímulos radiativos con base en las condiciones fisiológicas y físicas que podrían influir de manera realista en el proceso. 
			\paragraph{Fase 5: Contraste de los resultados teóricos y simulados con los datos observados en clínica y experimentos} Con los datos de la simulación, se realizará una comparación de los hallazgos observacionales en las prácticas clínicas y laboratoristas registras en la documentación científica actual. 
			
			\bibliographystyle{apalike}
			\nocite{Bleehen1988}
			\nocite{Hall2000}
			\nocite{Saha2006}
			\nocite{Tubiana1990}
			\nocite{CCM2637}
			\nocite{Redondo2014}
			\nocite{Moro2022}
			\bibliography{biblio_proyecto_seminario_rads.bib}
			

\end{document}